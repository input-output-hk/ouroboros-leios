\chapter{Mini Protocols}
\label{chapter:mini-protocols}

\newcommand{\Producer}{\textcolor{mygreen}{\textbf{Producer}}}
\newcommand{\Consumer}{\textcolor{myblue}{\textbf{Consumer}}}

\newcommand{\miniProtocolsUrl}{\url{https://ouroboros-network.cardano.intersectmbo.org/pdfs/network-spec/network-spec.pdf\#chapter.3}}
\section{Mini Protocols}
For background information on mini protocols see sections 3.1-3.4 of
the \emph{Ouroboros NetworkSpecification}\footnote{\miniProtocolsUrl},
and rest of that chapter for the mini protocols already in use. Here
we will present the additional node-to-node mini protocols needed for
Leios.

\newcommand{\miniEntry}[5]{
  \begin{framed}
      \noindent\textbf{#1}\hfill  Section \ref{#2}
      \newline {#3}
      \newline {\href{#5}{\small\texttt{#4}}}
  \end{framed}
}

\newcommand{\relay}{\text{Relay}}
\newcommand{\catchup}{\text{CatchUp}}
\newcommand{\fetch}{\text{Fetch}}

\section{\relay{} mini-protocol}
\label{ptcl:relay}

\newcommand{\StInit}  {\state{StInit}}
\newcommand{\MsgInit} {\msg{MsgInit}}
\newcommand{\StIdsBlocking}    {\state{StIdsBlocking}}
\newcommand{\StIdsNonBlocking} {\state{StIdsNonBlocking}}
\newcommand{\StData}              {\state{StData}}
\newcommand{\MsgRequestIdsNB}  {\msg{MsgRequestIdsNonBlocking}}
\newcommand{\MsgRequestIdsB}   {\msg{MsgRequestIdsBlocking}}
\newcommand{\MsgReplyIds}      {\msg{MsgReplyIds}}
\newcommand{\MsgReplyIdsAndAnns}  {\msg{MsgReplyIdsAndAnns}}
\newcommand{\MsgReplyIdsSlashAnns}  {\msg{MsgReplyIds\langle AndAnns\rangle}}
\newcommand{\MsgRequestData}      {\msg{MsgRequestData}}
\newcommand{\MsgReplyData}        {\msg{MsgReplyData}}
\newcommand{\option}[1]{\text{#1}}
\newcommand{\BoundedWindow} {\option{BoundedWindow}}
\newcommand{\Announcements} {\option{Announcements}}
\newcommand{\annCond}{\text{Ann. Condition}}
\newcommand{\info}{\text{info}}
\newcommand{\ann}{\text{ann}}
\newcommand{\ack}{\text{ack}}
\newcommand{\req}{\text{req}}
\newcommand{\id}{\text{id}}
%% \newcommand{\ids}{\text{ids}}
\newcommand{\datum}{\text{datum}}
%% \newcommand{\datums}{\text{datums}}
\newcommand{\parameter}{\text{parameter}}

\subsubsection{Description}
The \relay protocol presented here is a generalization of the tx
submission protocol used to transfer transactions between full
nodes. We will use the term \emph{datum} as a generic term for the
payloads being diffused, which we expect to be generally small:
transactions, input block headers, endorse block headers\footnote{at
the network layer we split an endorse block into header and body,
where the latter contains the references to other blocks}, vote
bundles for a single pipeline and voter.

The protocol follows a pull-based strategy where the consumer asks for
new ids/datums and the producer sends them back. It is designed to be
suitable for a trustless setting where both sides need to guard
against resource consumption attacks from the other side.

\paragraph{Options and Parameters} A \relay{} instance is specified by these options and parameters
\begin{description}
\item [\BoundedWindow] The peers manage a bounded window (i.e. FIFO queue) of datums available to be requested. Useful for datums that are otherwise unbounded in number.
\item [\Announcements] Producers provide additional announcements about datums, when available.
\item [\id{}] Identifier for a datum to be relayed, used when requesting the datum.
\item [\info{}] Extra information provided with a new id.
\item [\datum{}] The datum itself.
\item [\annCond{} (If \Announcements)] Condition for announcement.
\item [\ann{} (If \Announcements)] Announcement representation.
\end{description}

\paragraph{Instances} \relay{} protocol instances are listed in Table~\ref{table:relay-instances}.
Tx-submission is further parameterized by the maximum window size
allowed. IB-relay, EB-relay, and Vote-relay are each parametrized by
the maximum age beyond which a datum is no longer
relayed\footnote{older EBs and IBs referenced by the blockchain can be
accessed from the \catchup{} mini protocol (Sec.~\ref{ptcl:catch-up}).}. IB-relay and
EB-relay rely on \Announcements{} to let consumers know when block bodies
are available as well. Consumers request IB and EB bodies through the corresponding
\fetch{} mini protocol (Sec.~\ref{ptcl:fetch}). For EB-relay we specify the \datum{} to be eb-header, by which we mean the constant size part of an Endorse block, while the references (to IBs and EBs) are considered the body.
\todo{For IB, EB, and Vote the \info{} could actually be unit as we do not need to apply prioritization to headers. However the slot might provide useful filtering, such as avoid downloading any more votes of a pipeline once we have a certificate for a seen EB.}

\begin{table}[h!]
\begin{tabular}{l l l l l l l l}
\header{instance} & \header{\BoundedWindow} & \header{\Announcements} & \header{\id{}} & \header{\info{}}
                  & \header{\datum} & \header{\annCond{}} & \header{\ann{}} \\\hline
Tx-submission & Yes & No  & txid & size & tx          & N/A            & N/A \\
IB-relay      & No  & Yes & hash & slot & ib-header   & body available & unit \\
EB-relay      & No  & Yes & hash & slot & eb-header   & body available & unit \\
Vote-relay    & No  & No  & hash & slot & vote-bundle & N/A            & N/A \\
\end{tabular}
\caption{\relay{} mini-protocol instances.}
\label{table:relay-instances}
\end{table}

\subsection{State machine}

\begin{figure}[h!]
\begin{tikzpicture}[->,shorten >=1pt,auto,node distance=4.5cm, semithick]
  \tikzstyle{every state}=[fill=red,draw=none,text=white]
  \node[state, mygreen, initial] (I) at (-4,  0) {\StInit};
  \node[state, myblue]           (A) at ( 0,  0) {\StIdle};
  \node[state]                   (B) at ( 9, -4) {\StDone};
  \node[state, mygreen]          (C) at ( 4, -4) {\StIdsBlocking};
  \node[state, mygreen]          (D) at (-4, -4) {\StIdsNonBlocking};
  \node[state, mygreen]          (E) at ( 0,  4) {\StData};
  \draw (I)  edge[above]                    node[above]{\MsgInit}                                                (A);
  \draw (C)  edge[above]                    node[below]{\MsgDone}                                                (B);
  \draw (A)  edge[left, bend left=45]       node[fill = white, anchor = center]{\MsgRequestIdsB}               (C);
  \draw (C)  edge[right, bend left=15]      node[fill = white, anchor = center, above = 2pt]{\MsgReplyIdsSlashAnns}     (A);
  \draw (D)  edge[right, bend left=45]      node[fill = white, anchor = center]{\MsgReplyIdsSlashAnns}                  (A);
  \draw (A)  edge[right, bend left=15]      node[fill = white, anchor = center, below = 2pt]{\MsgRequestIdsNB} (D);
  \draw (A)  edge[left, bend right=45]      node[fill = white, anchor = center, above = 2pt]{\MsgRequestData}     (E);
  \draw (E)  edge[right,bend right=45]      node[fill = white, anchor = center, below = 2pt]{\MsgReplyData}       (A);
\end{tikzpicture}
  \caption{State machine of the Relay mini-protocol.}
\label{fig:relay-automata}
\end{figure}

\begin{figure}[h]
  \begin{center}
    \begin{tabular}{l|l}
      \header{state}      & \header{agency} \\\hline
      \StInit             & \Producer \\
      \StIdle             & \Consumer \\
      \StIdsBlocking    & \Producer \\
      \StIdsNonBlocking & \Producer \\
      \StData              & \Producer \\
    \end{tabular}
    \caption{Relay state agencies}
  \end{center}
\end{figure}
\paragraph{Grammar}
\[
\begin{array}{l c l l}
\ack{} & ::= & \text{number} & \text{if}\,\BoundedWindow{}\\
       & |  & \text{unit} & \text{otherwise}\\
\req{} & ::= & \text{number} \\
\end{array}
\]
\paragraph{Protocol messages}
\begin{description}
\item [\MsgInit] initial message of the protocol
\item [\MsgRequestIdsNB{} {\boldmath $(\ack{},\req{})$}]
      The consumer asks for new ids and acknowledges old ids.
      The producer replies immediatly, possibly with an empty reply if nothing new is available.
\item [\MsgRequestIdsB{} {\boldmath $(\ack{},\req{})$}]
      Same as \MsgRequestIdsNB{} but the producer will block until the reply will be non-empty.
\item [\MsgReplyIds{} {\boldmath ($\langle (\id{}, \info) \rangle$) }]
      The producer replies with a list of available datums.
      The list contains pairs of ids and corresponding information about the identified datum.
      In the blocking case the reply is guaranteed to contain at least one id.
      In the non-blocking case, the reply may contain an empty list.
\item [\MsgReplyIdsAndAnns{} {($\langle (\id{}, \info{}) \rangle, \langle (\id{}, \ann{}) \rangle$) } (Requires \Announcements{})]
      Same as \MsgReplyIds{} but additionally the producer might, at most once, also
      provide an announcement for any id it has sent, in this message or
      previously.
\item [\MsgRequestData{} {\boldmath ($\langle \id{} \rangle$)}]
      The consumer requests datums by sending a list of ids.
\item [\MsgReplyData{} {($\langle \datum{} \rangle$)}]
      The producer replies with a list of the requested datums, some may be missing if no longer available for relay.
\item [\MsgDone{}]
      The producer terminates the mini protocol.
\end{description}

\begin{table}[h!]
  \begin{tabular}{l|l|l|l}
    \header{from state} & \header{message}    & \header{parameters}           & \header{to state}   \\\hline
    \StInit             & \MsgInit            &                               & \StIdle             \\
    \StIdle             & \MsgRequestIdsB   & $\ack{}$,$\req{}$                   & \StIdsBlocking    \\
    \StIdsBlocking    & \MsgReplyIds      & $\langle (\id{}, \info{}) \rangle$  & \StIdle             \\
    \StIdle             & \MsgRequestIdsNB  & $\ack{}$,$\req{}$                   & \StIdsNonBlocking \\
    \StIdsNonBlocking & \MsgReplyIds      & $\langle (\id{}, \info{}) \rangle$  & \StIdle             \\
    \StIdle             & \MsgRequestData      & $\langle \id{} \rangle$         & \StData              \\
    \StData              & \MsgReplyData        & $\langle \datum{} \rangle$         & \StIdle             \\
    \StIdsBlocking   & \MsgDone            &                               & \StDone             \\
\multicolumn{4}{l}{If \Announcements{} is set, \MsgReplyIds{} messages are replaced with \MsgReplyIdsAndAnns{}:}\\
    \StIdsBlocking    & \MsgReplyIdsAndAnns & $\langle (\id{}, \info{}) \rangle$,$\langle (\id{}, \ann{}) \rangle$   & \StIdle             \\
    \StIdsNonBlocking    & \MsgReplyIdsAndAnns & $\langle (\id{}, \info{}) \rangle$,$\langle (\id{}, \ann{}) \rangle$   & \StIdle             \\
  \end{tabular}
  \caption{\relay{} mini-protocol messages.}
  \label{table:relay-messages}
\end{table}

\iffalse
\subsection{Size limits per state}

Table~\ref{table:tx-submission-size-limits} specifies how may bytes can be send
in a given state, indirectly this limits payload size of each message.  If
a space limit is violated the connection SHOULD be torn down.

\begin{table}[h!]
  \begin{center}
    \begin{tabular}{l|r}
      \header{state}      & \header{size limit in bytes} \\\hline
      \StInit             & \texttt{5760} \\
      \StIdle             & \texttt{5760} \\
      \StIdsBlocking    & \texttt{2500000} \\
      \StIdsNonBlocking & \texttt{2500000} \\
      \StData              & \texttt{2500000} \\
    \end{tabular}
    \caption{size limits per state}
    \label{table:tx-submission-size-limits}
  \end{center}
\end{table}

\subsection{Timeouts per state}

The table~\ref{table:tx-submission-timeouts} specify message timeouts in
a given state.  If a timeout is violated the connection SHOULD be torn down.

\begin{table}[h!]
  \begin{center}
    \begin{tabular}{l|r}
      \header{state}      & \header{timeout} \\\hline
      \StInit             & - \\
      \StIdle             & - \\
      \StIdsBlocking    & - \\
      \StIdsNonBlocking & \texttt{10}s \\
      \StData              & \texttt{10}s \\
    \end{tabular}
    \caption{timeouts per state}
    \label{table:tx-submission-timeouts}
  \end{center}
\end{table}

\subsection{CDDL encoding specification}\label{tx-submission2-cddl}
\lstinputlisting[style=cddl]{tx-submission2.cddl}
\fi

\subsection{Producer and Consumer Implementation}
The protocol has two design goals: It must diffuse datums with high efficiency
and, at the same time, it must rule out
asymmetric resource attacks from the consumer against the provider.

The protocol is based on two pull-based operations.
The consumer can ask for a number of ids and it can use these
ids to request a batch of datums.
The consumer has flexibility in the number of ids it requests,
whether to actually download the datum of a given id
and flexibility in how it batches the download of datums.
The consumer can also switch between requesting ids and downloading
datums at any time. The protocol supports blocking and non-blocking requests for new ids.
The producer must reply immediately (i.e. within a small timeout) to a non-blocking request.
It replies with not more than the requested number of ids (possibly with an empty list).
A blocking request on the other hand, waits until at least one datum is available.

It must however observe several constraints that are necessary for a
memory efficient implementation of the provider.

\paragraph{With \BoundedWindow{}}
Conceptually, the provider maintains a limited size FIFO of outstanding transactions per consumer.
(The actual implementation can of course use the data structure that works best).
The maximum FIFO size is a protocol parameter.
The protocol guarantees that, at any time, the consumer and producer agree on the current size of
that FIFO and on the outstanding transaction ids.
The consumer can use a variety of heuristics for requesting transaction ids and transactions.
One possible implementation for a consumer is to maintain a FIFO which mirrors the producer's FIFO
but only contains the \id{} and \info{} pairs and not the \datum{}.
%
After the consumer requests new ids, the provider replies with a list of ids and
puts these datums in its FIFO. If the FIFO is empty the consumer must use a blocking
request otherwise a non-blocking request.
As part of a request a consumer also acknowledges the number of old datums,
which are removed from the FIFO at the same time.
The provider checks that the size of the FIFO, i.e. the number of outstanding datums,
never exceeds the protocol limit and aborts the connection if a request violates the limit.
The consumer can request any batch of datums from the current FIFO in any order.
Note however, that the reply will omit any datums that have become invalid in the meantime.
(More precisely the producer will omit invalid datums from the reply but they will still be counted in the FIFO
size and they still require an acknowledgement from the consumer).

\paragraph{Without \BoundedWindow{}} A \relay{} mini protocol instance that does not make use of \BoundedWindow{} will want to rely on other ways to bound the amount of datums that can be requested at a given time.
The consumer shall request ids in a blocking way when it does not
intend on requesting any of the available datums.


\paragraph{Equivocation handling} IB-relay, EB-relay, and Vote-relay must guard against the possibility of equivocations, i.e. the reuse of a generation opportunity for multiple different blocks.
The \emph{message identifier} of an header is the pair of its
generating node id and the slot it was generated for\footnote{for IBs/EBs also its subslot, in
case generation frequency is greater than $1/\text{slot}$}. Two headers
with the same message identifier constitute a \emph{proof of
equivocation}, and the first header received with a given message
identifier is the \emph{preferred header}. For headers with the same
message identifier, only the first two should be relayed, furthermore
only the body of the preferred header should be fetched.

\section{\fetch{} mini-protocol}
\label{ptcl:fetch}

\renewcommand{\StIdle}{\state{StIdle}}
\renewcommand{\StBusy}{\state{StBusy}}
\newcommand{\StStreaming}{\state{StStreaming}}
\renewcommand{\StDone}{\state{StDone}}
\newcommand{\MsgRequestBodies}{\msg{MsgRequestBodies}}
\newcommand{\MsgStartBatch}{\msg{MsgStartBatch}}
\newcommand{\MsgNoBlocks}{\msg{MsgNoBlocks}}
\newcommand{\MsgBody}{\msg{MsgBody}}
\newcommand{\MsgBatchDone}{\msg{MsgBatchDone}}
\newcommand{\MsgConsumerDone}{\msg{MsgConsumerDone}}
\newcommand{\point}{\text{point}} \newcommand{\slot}{\text{slot}}
\newcommand{\body}{\text{body}} \newcommand{\request}{\text{request}}
\newcommand{\hash}{\text{hash}} \newcommand{\rbrange}{\text{range}}
\subsection{Description}

The \fetch{} mini protocol enables a node to download block bodies.
It is a generalization of the BlockFetch mini protocol used for
base blocks: IBs and EBs do not have a notion of range, so they are
requested by individual identifiers.
\todo{Generalizing from BlockFetch means we deliver bodies in a streaming fashion, is that appropriate for IBs and EBs?}

\paragraph{Parameters} A \fetch{} instance is specified by these parameters
\begin{description}
\item [\request{}] request format for a sequence of blocks.
\item [\body{}] Block body itself.
\end{description}

\paragraph{Instances} \fetch{} instances are listed in Table~{table:fetch-instances}. The \body{} descriptions included here are for illustration, in particular to clarify what we mean by \body{} of an Endorse block. A \point{} is a pair of \slot{} and \hash{}, the \slot{} allows for better indexing. A \rbrange{} is a pair of two of $\point{} \mid \text{origin}$.
The IB-fetch and EB-fetch instances are intended for on-the-fly block diffusion, complementing the corresponding \relay{} mini protocols.
\begin{table}[h!]
\begin{tabular}{l l l l l l l l}
\header{instance} &  \header{\request{}} & \header{\body{}} \\\hline
IB-fetch  & $[\point{}]$ & $[\text{Tx}]$ \\
EB-fetch  & $[\point{}]$ & $([\text{IBRef}],[\text{EBRef}])$\\
BlockFetch  & $\rbrange{}$ & $\text{RB body}$\\
\end{tabular}
\caption{\fetch{} mini-protocol instances.}
\label{table:fetch-instances}
\end{table}

\subsection{State machine}

\begin{figure}[h]
  \begin{tikzpicture}[->,shorten >=1pt,auto,node distance=4.5cm, semithick]
    \tikzstyle{every state}=[fill=red,draw=none,text=white]
    \node[state, mygreen,initial] at (-1cm,0cm)  (Idle)      {\StIdle};
    \node[state]                  at (4cm,0cm)   (Done)      {\StDone};
    \node[state, myblue]          at (-3cm,-3cm) (Busy)      {\StBusy};
    \node[state, myblue]          at (4cm,-3cm)  (Streaming) {\StStreaming};

    \draw (Idle)      edge[above]            node[fill=white]{\MsgConsumerDone}           (Done);
    \draw (Idle)      edge[left,bend right]  node[fill=white]{\MsgRequestBodies}         (Busy);
    \draw (Busy)      edge[above,bend right] node[fill=white]{\MsgNoBlocks}             (Idle);
    \draw (Busy)      edge[below]            node[fill=white]{\MsgStartBatch}           (Streaming);
    \draw (Streaming) edge[loop right]       node[fill=white,left=-10mm]{\MsgBody}     (Streaming);
    \draw (Streaming) edge[right]            node[fill=white,left=-15mm]{\MsgBatchDone} (Idle);
  \end{tikzpicture}
  \caption{State machine of the \fetch{} mini-protocol}
\end{figure}

\begin{figure}[h]
  \begin{center}
    \begin{tabular}{l|l}
      \header{state} & \header{agency} \\\hline
      \StIdle        & \Consumer \\
      \StBusy        & \Producer \\
      \StStreaming   & \Producer \\
    \end{tabular}
    \caption{\fetch{} state agencies}
  \end{center}
\end{figure}

\paragraph{Protocol messages}
\begin{description}
\item [\MsgRequestBodies{} {\boldmath $\request{}$}]
  The consumer requests a sequence of bodies from the producer.
\item [\MsgNoBlocks]
  The producer tells the consumer that it does not have all of the blocks in the requested sequence.
\item [\MsgStartBatch]
  The producer starts body streaming.
\item [\MsgBody{} {\boldmath $(\body{})$}]
  Stream a single block's body.
\item [\MsgBatchDone]
  The producer ends block streaming.
\item [\MsgConsumerDone]
  The consumer terminates the protocol.
\end{description}

Transition table is shown in table~\ref{table:fetch}.

\begin{table}[h!]
  \begin{center}
    \begin{tabular}{l|l|l|l}
      \header{from state} & \header{message} & \header{parameters} & \header{to state} \\\hline
      \StIdle       & \MsgConsumerDone   &            & \StDone      \\
      \StIdle       & \MsgRequestBodies & $\request{}$    & \StBusy      \\
      \StBusy       & \MsgNoBlocks     &            & \StIdle      \\
      \StBusy       & \MsgStartBatch   &            & \StStreaming \\
      \StStreaming  & \MsgBody        & $\body{}$     & \StStreaming \\
      \StStreaming  & \MsgBatchDone    &            & \StIdle      \\
    \end{tabular}
  \end{center}
  \caption{\fetch{} mini-protocol messages.}
  \label{table:fetch}
\end{table}

\iffalse
\subsection{Size limits per state}

These limits bound how many bytes can be send in a given state, indirectly this
limits payload size of each message.  If a space limit is violated the
connection SHOULD be torn down.

\begin{table}[h!]
  \begin{center}
    \begin{tabular}{l|r}
      \header{state} & \header{size limit in bytes} \\\hline
      \StIdle        & \texttt{65535} \\
      \StBusy        & \texttt{65535} \\
      \StStreaming   & \texttt{2500000} \\
    \end{tabular}
    % \caption{size limits per state}
  \end{center}
\end{table}

\subsection{Timeouts per state}

These limits bound how much time the receiver side can wait for arrival of
a message.  If a timeout is violated the connection SHOULD be torn down.

\begin{table}[h!]
  \begin{center}
    \begin{tabular}{l|r}
      \header{state} & \header{timeout} \\\hline
      \StIdle        & - \\
      \StBusy        & \texttt{60}s \\
      \StStreaming   & \texttt{60}s \\
    \end{tabular}
    % \caption{timeouts per state}
  \end{center}
\end{table}
\fi

\paragraph{Implementation}
The high-level description of the Leios protocol specifies
freshest-first delivery for IB bodies, to circumvent attacks where a
large amount of old IBs gets released by an adversary. The \relay{}
mini protocol already takes a parameter that specifies which IBs are
still new enough to be diffused, so older IBs are already
deprioritized to only be accessible through the \catchup{} protocol,
and only if referenced by other blocks.
%
Nevertheless consumers should take care to send approximately just enough body requests to utilize the available bandwidth, so that they have more choices, and more up to date information, when deciding which blocks to request from which peers.

\section{\catchup{} mini-protocol}
\label{ptcl:catch-up}

\subsection{Description}
The \catchup{} mini protocol allows for nodes to obtain IB and EB
blocks referenced by the chain. These will typically be too old to be
diffused by the \relay{} and \fetch{} mini protocols, but are still
relevant to reconstruct the ledger state. Additionally it covers
certified EBs not yet in the chain but which are still recent enough
for inclusion in a future ranking block, and any blocks they
reference.
%
\todo{Unless we specify recent certified EBs are to be offered through
  the \relay{} protocol still, in which case request
  \ref{catchup:req:recent-ebs} can be dropped.}
%
This data, together with the base chain, is what is needed
for a node to participate in future pipelines.

The protocol should allow the consumer to divide the requests between
different producers, and for the producer to have an efficient way to
retrieve the requested blocks.
%
The consumer should be able to retrieve the base chain through the
other mini protocols, and so the EB references within. However, the
slots of those EBs are unknown, as well as any indirect references.

\paragraph{Requests}
\begin{description}
\item[EBs by RB \rbrange{}] given an RB \rbrange{} from its chain, the producer
  should reply with all EBs which are (i) transitively referenced by RBs in that
  range, (ii) not referenced by earlier RBs.
\item[Recent certified EBs by \slot{} range]\label{catchup:req:recent-ebs} given a slot range, the
  producer should reply with all certified EBs which are (i) generated
  in the slot range, (ii) not referenced by RBs\footnote{Restriction
  (ii) is to avoid overlap with an RB range query, but could be dropped to save on complexity if not worth the saved bandwidth}. The start of the
  slot range should be no earlier than the oldest slot an EB could be
  generated in and still referenced in a future RB.
\item[Certificate by EB \point{}] given the \point{} of a certified EB not referenced by the chain, the producer should reply with a certificate for it. Needed for inclusion of the EB into a future RB produced by the consumer.
\item[IBs by EB \point{}, and \slot{} range] given a \point{} for a
  certified EB, the producer should reply with all the IBs which are (i)
  generated in the given slot range, (ii) directly referenced by
  the EB. The slot range allows for partitioning request about the
  same EB across different peers.
\end{description}
\todo{The \emph{IBs by EB \point{}, and \slot{} range} request could be replaced by just a list of IB \point{}s, if IB references in EB bodies are augmented with the IB slot. Maybe size of request could become a consideration: by EB \point{} and \slot{} range the request size is $56$ bytes for possibly all the referenced IBs at once, while by IB \point{} the size is $40$ bytes each, and there could be double digits of them. If expect to always fragment requests to just a few IBs at a time the difference is perhaps not important.}
\todo{The \emph{EBs by RB \rbrange{}} request could similarly be replaced by a list of EB \point{}s,
if EB references in RBs and EBs are augmented with the EB slot. In this case, the consumer would be in charge of discovering needed referenced EBs as it fetches the ones it knows about.}

\paragraph{Definition} The \catchup{} protocol is defined as a new instance of the \fetch{} protocol. We give the parameters as a grammar
\[
\begin{array}{l c l l}
\request{}_{\catchup{}} & ::= & \text{ebs-by-rb-range}(\rbrange{})\\
       & \mid  & \text{ebs-by-slot-range}(\slot{},\slot{})\\
       & \mid  & \text{ibs-by-eb-and-slot-range}(\point{},(\slot{},\slot{}))\\
\body{}_{\catchup{}} & ::= & \text{ib-block}(\text{ib-header},\text{ib-body}) \\
        & \mid & \text{eb-block}(\text{eb-header}, \text{eb-body})\\
        & \mid & \text{eb-certificate}(\text{certificate})\\
\end{array}
\]
alternatively there could be separate mini protocols for IB, EB, and Certificate \catchup{}, so that there cannot be a format mismatch between requests and replies.

\paragraph{Implementation} To fulfill the higher-level freshest-first delivery goal, we might need to stipulate that producers should prioritize serving requests for the \{IB,EB,Vote\}-\relay{} and \{IB,EB\}-\fetch{} mini protocols over requests for \catchup{}.
